% Chapter 1

\chapter{Introducción general} % Main chapter title

\label{Chapter1} % For referencing the chapter elsewhere, use \ref{Chapter1} 
\label{IntroGeneral}

%----------------------------------------------------------------------------------------

% Define some commands to keep the formatting separated from the content 
\newcommand{\keyword}[1]{\textbf{#1}}
\newcommand{\tabhead}[1]{\textbf{#1}}
\newcommand{\code}[1]{\texttt{#1}}
\newcommand{\file}[1]{\texttt{\bfseries#1}}
\newcommand{\option}[1]{\texttt{\itshape#1}}
\newcommand{\grados}{$^{\circ}$}

%----------------------------------------------------------------------------------------

En este capítulo se explican las causas que dan origen al dispositivo desarrollado y la función que este cumple en el proyecto de la empresa Skyloom Global. Se hace una breve descripción de un amplificador óptico, del hardware ya existente y por último se presentan los objetivos y alcances determinados durante la planificación de este trabajo.


%----------------------------------------------------------------------------------------
\section{Contexto de la empresa}
\label{sec:contexto}

La creciente demanda de mayores velocidades de transmisión de información en el ámbito aeroespacial, combinada con las limitaciones de la radiofrecuencia, favorecen el surgimiento de nuevas empresas enfocadas en el desarrollo de nuevas tecnologías capaces de superar estas barreras. Tal es el caso de la empresa Skyloom Global, que propone la creación de una red de satélites de órbita baja (LEO, del inglés, Low Earth Orbit) \citep{WEBSITE:LEO} utilizando enlaces ópticos de alta velocidad.

Esta red se basa en un satélite geoestacionario (órbita GEO) \citep{WEBSITE:GEO} al que los satélites en LEO transmiten la información (denominado \textit{uplink}) para luego descargar esta a una estación terrena (denominado \textit{downlink}). Como los satélites en LEO completan unas 11 vueltas sobre su órbita por día, no tienen una visión constante de la estación terrena, sino que solo están en contacto con ella durante determinadas ventanas de tiempo. Por lo tanto, si se quisiera descargar datos, esto solo sería posible durante esas ventanas, lo que genera un retraso considerable entre el momento en que el satélite obtiene la información y el momento en que se descarga a Tierra. 

El satélite en GEO, en cambio, al tener el mismo período orbital que la Tierra, mantiene una visión continua de la estación terrena. De esta forma, al realizar el traspaso de información, se pueden obtener los datos con alta disponibilidad y mínimo retraso. Un diagrama de esta propuesta se muestra en la figura \ref{fig:propSky}.

\begin{figure}[H]
\centering
\includegraphics[width=0.8\textwidth]{./Figures/propuesta_skyloom.png}
\caption{Propuesta de la empresa Skyloom Global\protect\footnotemark.}
\label{fig:propSky}
\end{figure}

\footnotetext{Imagen tomada de \url{https://www.skyloom.co/}}

El enlace óptico, que permite la transmisión y recepción de datos a una velocidad de 1 Gbps (1 \textit{Gigabit} por segundo) entre satélites, se establece mediante una terminal de comunicaciones óptica, que es el principal producto actualmente en desarrollo por la empresa. Estas terminales cuentan con un láser que trabaja sobre una longitud de onda de 1550 nm (banda C del espectro). Dicho láser es el encargado de transmitir la información propiamente dicha mediante pulsos de luz (no visible).

En líneas generales, los láseres de esta longitud de onda que se encuentran en el mercado no cuentan con la potencia óptica necesaria para que sean detectados por el receptor. Esto se debe principalmente a que la distancia de espacio libre estimada entre dos satélites en LEO es de unos 4000 km \citep{WEBSITE_SKY}.

Para solucionar el problema antes mencionado, se introduce a la salida del láser un amplificador dopado con erbio o EDFA (del inglés, \textit{Erbuim Doped Fiber Amplifier}) \citep{WEBSITE:EDFA2}. La función de este dispositivo es aumentar la potencia del láser varias veces, de forma que se alcance un nivel adecuado para la transmisión.

El modelo de amplificador utilizado por la empresa cuenta con un conector que provee una interfaz electrónica para poder controlar su funcionamiento. Esta interfaz cuenta con distintas señales y buses de comunicación que se explican con mayor detalle en la sección \ref{sec:intAmp}.

Este amplificador formará parte de la terminal de comunicaciones y estará sometido a intensivas pruebas de funcionamiento y rendimiento durante la etapa de investigación y desarrollo. Para esto, es indispensable contar con una herramienta que permita a los ingenieros a cargo utilizar el amplificador de forma aislada, es decir, sin depender del hardware del producto final.

En la figura \ref{fig:bloquesProy} se puede ver un esquema de uso del dispositivo y sus conexiones con el hardware externo.

\begin{figure}[H]
\centering
\includegraphics[width=0.85\textwidth]{./Figures/bloquesProy.png}
\caption{Esquema de uso y conexionado del sistema.}
\label{fig:bloquesProy}
\end{figure}

El dispositivo cuenta con tres conexiones externas: la interfaz con el amplificador, la conexión a la fuente de alimentación y un puerto USB.

La interfaz con el EDFA permite controlarlo y consultar diversos parámetros de funcionamiento a través de las señales presentes en el conector. La fuente de alimentación se encarga de energizar el tester y el EDFA. Y por último, la conexión USB permite controlar el amplificador del mismo modo que en el tester, por lo que el uso de la PC es opcional. Para poder hacer esto, sobre esta debe correr un software que permita establecer una comunicación.

%----------------------------------------------------------------------------------------

\section{Estado del arte}

Si bien hoy en día los EDFA son ampliamente utilizados en varias aplicaciones de optoelectrónica, en la actualidad no existe un estándar para la interfaz eléctrica que poseen estos amplificadores. Esto lleva a que cada fabricante defina una propia para sus productos y que el usuario tenga que adaptar su hardware para poder usarlos.

Para el caso en que se necesite contar con un tester o una placa de depuración para hacer uso o pruebas sobre el dispositivo, el usuario debe adquirir una del mismo fabricante del amplificador por esta misma razón. Actualmente, la empresa hace uso de una de estas placas, la cual se puede ver en la figura \ref{fig:placaDebug}.

\begin{figure}[H]
\centering
\includegraphics[width=0.60\textwidth]{./Figures/placaDebug.jpg}
\caption{Placa de depuración.}
\label{fig:placaDebug}
\end{figure}

Esta placa, denominada EDFA IFC (Interface Card), cuenta con la electrónica necesaria para proveer al usuario algunas funcionalidades básicas que le permiten hacer uso del amplificador. Contiene una interfaz USB-UART para usarse junto con una computadora, LEDs para las señales de alarmas y conectores para la alimentación del EDFA.

\section{Motivación}

La empresa ya lleva algún tiempo haciendo uso de la placa mecionada en la sección anterior y recientemente ha tomado la decisión de dejar de usarla y diseñar una propia debido a los motivos que se listan a continuación:

\begin{itemize}
\item No permite el uso de todas las señales presentes en la interfaz del amplificador.
\item Tiene una tasa de fallas muy alta, en particular la interfaz UART.
\item Su costo es muy elevado en relación a sus prestaciones.
\item El tiempo de entrega del producto es de varias semanas.
\item No posee las protecciones eléctricas necesarias para proteger el amplificador.
\end{itemize}

Las desventajas aquí expuestas son los principales motivos que dieron origen a la necesidad de contar con el sistema propuesto en este trabajo. Esto le permite a la empresa no depender del fabricante para la entrega de estos testers y por lo tanto reducir costos y tiempos.

Por otro lado, el tester desarrollado en este trabajo no solo posee las mismas funcionalidades que el ofrecido por el fabricante sino que también incorpora otras características adicionales que lo hacen más fácil de usar, menos propenso a fallas y más seguro.

%----------------------------------------------------------------------------------------

\section{Objetivos y alcance}

El objetivo principal de este trabajo fue el desarrollo de un dispositivo capaz de controlar y realizar mediciones sobre un amplificador de fibra óptica. Las tareas contempladas fueron:

\begin{itemize}
\item Diseño y construcción de un prototipo funcional del dispositivo.
\item Diseño e implementación del firmware del dispositivo.
\item Diseño de los bancos de prueba y ensayos.
\item Simulación del funcionamiento del hardware mediante software.
\item Documentación de diseño y manual de uso.
\end{itemize}

Los puntos del desarrollo que no se contemplaron en el trabajo fueron:

\begin{itemize}
\item Diseño y construcción de la versión final del dispositivo.
\item Especificación de las pruebas a ejecutar sobre el amplificador óptico utilizando el dispositivo.
\item Fabricación del PCB del dispositivo.
\item Procesamiento e interpretación de los valores de los parámetros del EDFA.
\item Diseño y construcción de la fuente de alimentación externa.
\item Diseño e implementación del software a ejecutarse en la PC.
\end{itemize}


%----------------------------------------------------------------------------------------
